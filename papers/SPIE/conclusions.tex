\section{Conclusions}

We presented methods to explore a social network communication graph by topic and community.  Our procedure starts with a topic and builds up a community seeded by a pair of the most active communicators on the topic and their mutual friends (triangles), added recursively.  A fringe of such a community represents people related to those organized around the topic, but also pointing to other topics of interest.  We used n-gram analysis and statistically improbable phrases (SIPs) to characterize the topics of interest in a community and its fringe.  The SIPs allow to select communities of interest or further topics for browsing and community building, thus enabling community sensing by pivoting over SIPs.  Our workflow provides an effective way to explore a vast social network in an iterative way.	 When using topics focused temporally or geographically, fine-grained community sensing becomes possible.

