\subsection{Topics}

For the communities we identify, we look at the typical topics discussed.  One way to find the topics is to look at the top \emph{n-grams} \cite{Ngram:Chen-Goodman} used in the text of the messages exchanged.  These will often be trivial and shared across many communities, e.g. bigrams such as ``I need...''  A more interesting source for characteristic topics are the Statistically Improbable Phrases, or SIPs.  They are widely used on Amazon.com to characterize a book.  The SIPs are found by looking at n-grams and comparing their actual frequency to the case where the words would happen to be together randomly.  Those n-grams where such repeated co-occurrence is very unlikely for random words are ``improbable'' in terms of randomness, i.e. they are uncovering some regularity.  Some of the Glenn Beck community SIPs are shown in Table~\ref{table:gb-sips}.

\begin{table}
\label{table:gb-sips}
	\centering
\begin{tabular}{|c|c|}
	\hline
	Reining Medical Suits & Freedoms First Rally \\
	\hline
	Voting Rights Act & Martin Luther King \\
	\hline
	Gitmo Detainee Lawyer & Convo Gitmo Detainee \\
	\hline
	Able Bodied One & Payer Health Care \\
	\hline
	Speaking Your Truth & Town Hall Friday \\
	\hline
	Henry Louis Gates & Gov Mark Sanford \\
	\hline 
\end{tabular}
\caption{Statistically Improbable Phrases from the Glenn Beck community.  The phrases are trigrams which co-occur much more than random three words would.  They are the political issues and personalities of the day, circa Summer of 2009.}
\end{table}


\subsection{Applications}

Our method is a convenient way to browse through a social network.  The Twitter data stream receives millions of messages from millions of users daily.  It's hard to even begin to explore such a collection.  Given the topical community search, we have both an entry point and a pivoting procedure.  After a community is seeded, grown, and its topics are revealed via SIPs, the explorer can either stop there and look at the users of interest, or detect the fringe and pick a topic from its SIPs or a pair with one member in the community at hand and another a fringe member.  That pair, or a fringe SIP can be used to seed a new community for the next step of browsing, which we call pivoting.

When keywords contain spatially or temporally localized topics, our workflow can be used for community sensing -- following the communities in a specific environment in the topical or geographical space.  For instance, the 2009 crisis in the New York State Senate can be tracked down using the name of a key Democratic senator, Pedro Espada Jr., who sided with the Republicans for a while, only to come back having extracted certain concessions -- and bringing the senate business to a halt in the mean time, causing much Twitter debate. Those users who mention his name are likely to follow the state politics; similar topics exist for engaging issues in every locale.
