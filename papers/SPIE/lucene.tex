\section{Text Indexing}

For topical search, we index the twits and search them with Lucene \cite{code:lucene}.
Lucene is a powerful open-source search engine which uses flat-file inverted indices to provide fast full-text indexing and searching.  The fundamental unit in Lucene is a document, which can have any number of fields, each with a number of indexing parameters.

We created a custom analyzer for Lucene allowing us to preserve the meaningful special characters that appear in twits.  The indexer tokenizes the text on characters which are neither letters, numbers or the @ or \# symbols and creates a full-text search index on those tokens.  We also store metadata for each twit, including the date the twit was posted to the twitter stream, whether the twit contains a hashtag, is it directed, and, if so, who it was directed to.

We created two Lucene indices on a month's worth of twits from the Twitter \emph{gardenhose} stream, containing a statistically representative fraction of all twits (which can be as high as a fourth).  The first index was created with each twit as a document.  A second index was then created on each user's corpora as a document.  Using these two indices we created a number of API functions to perform textual analysis on the Twitter data.

\subsection{Full Index API Functions}

In order to ensure that we would receive meaningful results from the analysis of the twitter data we created a custom stop word list generator which would generate a stop-word list based on the whole corpora of words used in twits, pruning out some entries by hand that were clearly meaningful.  We created a number of functions which performed textual analysis on the dataset.

\begin{enumerate}
\item \emph{topChatters} - gets the users with the most conversation messages
\item \emph{twitUserIds} - gets all the users and twits for a term
\item \emph{topWordsByTwit} - gets the top words in the combined corpora of the list of twits
\item \emph{topWordsByUser} - gets the top words in the combined corpora of the list of users
\item \emph{userPairTopics} - gets the top words for pairs of users
\item \emph{getStopList} - creates a stop list from the most commonly used words in the corpora of all twits
\end{enumerate}

\subsection{User Corpora Index API Functions}
In addition to the API functions which interact with the full index on a per twit basis we created API functions which operate on the per user corpora index.  These functions primarily use queries with very large numbers of terms to perform cosine similarity analysis on the corpora of a single user or a group of users against another user or group of users.  Through this procedure we are able to compare a candidate to join a group to the corpora of the group as a whole in order to rank users who are considered to be on the fringe of a community for inclusion into the community.
