\section{Introduction}

Twitter is a microblogging site and social network.  Users of Twitter post or ``tweet'' short (up to 140 character) messages.  Users can follow others tweets and interact with others using the directed messages (denoted by @) and hashtags (denoted by \#). Twitter provides a conventional social graph of news updates.  However, due to its ubiquity, Twitter can also be treated as a sensor.  Many individuals update their status often enough to provide background activity.  For brevity, we refer to a message on Twitter as a \emph{twit}, and a twit mentioning another user as a \emph{reply}.

Unlike many other social networks, Twitter does not provide any kind of explicit structure for community or group organization.  As such, any notional community must be constructed solely from the directionality of the messages on the network and the content of those messages.  Any such community is necessarily \emph{ad hoc} and dynamic, as a user can connect to many people over time and talk about a variety of topics.
