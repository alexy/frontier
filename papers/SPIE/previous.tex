\section{Previous Work}

Community identification in graphs with no specific community marking is an active research area.  Flake et al. \cite{DBLP:conf/kdd/FlakeLG00} propose a definition of a community as a group with more links inside than outside.  It turns out such a definition requires a simultaneous partitioning of the whole graph into communities to work, and the communities cannot overlap.  Mishra et al. \cite{DBLP:conf/waw/MishraSST07} overcome this with a notion of $(\alpha,\beta)$ clusters.  A subset of vertices forms an $(\alpha,\beta)$-cluster if every vertex in the cluster is adjacent to at least a $\beta$-fraction of the cluster and every vertex outside the cluster is adjacent to at most an $\alpha$-fraction of the cluster.  Such clusters can overlap.  Backstrom et al. \cite{DBLP:conf/kdd/BackstromHKL06} use self-declaration as criteria for membership: joining a LiveJournal community, which requires explicit joining, or publishing in a conference for DBLP-based communities.  Statistically Improbable Phrases (SIPs) are effectively used on Amazon to characterize contents of books, and they give a very good idea of what a book is about.
